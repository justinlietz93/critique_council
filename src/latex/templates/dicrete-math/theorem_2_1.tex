
\documentclass[12pt,reqno]{amsart}

\usepackage{graphicx}

\usepackage{amssymb}
\usepackage{amsthm}
\theoremstyle{plain}

\newtheorem*{theorem*}{Theorem}
%% this allows for theorems which are not automatically numbered

\newtheorem{definition}{Definition}
\newtheorem{theorem}{Theorem}
\newtheorem{lemma}{Lemma}
\newtheorem{example}{Example}
\usepackage{lineno}

%% The above lines are for formatting.  In general, you will not want to change these.


\title{Theorem $2.1$}

\author{Adrienne Stanley}

\begin{document}

\begin{abstract}
We prove theorem $2.1$ using the method of proof by way of contradiction.  This theorem states that for any set $A$, that in fact the empty set is a subset of $A$, that is $\emptyset \subset A$.
\end{abstract}

\maketitle

We first start with a discussion of subsets.

\begin{definition}
Let $A$ and $B$ be sets.  We say $A$ is a subset of $B$ if every element in $A$ is also an element of $B$ and we write $A \subset B$.  This can also be written as
$$ (A \subset B) \leftrightarrow \forall x ( x \in A \to x \in B).$$
\end{definition}

Notice that for sets $A$ and $B$, if $A \not\subset B$, then there exists an element $x$ such that $x \in A$ and $x \notin B$.  That is,
$$(A \not\subset B) \leftrightarrow \exists x ( x \in A \wedge x \notin B).$$

\begin{example}
Let $A = \{ 1, 2, 3, 4, 5 \} $, $B = \{ 1, 2 \}$ and $C = \{ 1, 7 \}$.  We can see that every element in $B$ is an element of $A$.  Further, we can see that $C$ contains an element, namely $7$, which is not in $A$.  Thus, $B \subset A$ and $C \not\subset A.$
\end{example}

We now prove theorem $2.1$.

\begin{theorem*}[2.1]
For any set $A$, $\emptyset \subset A$.
\end{theorem*}

\begin{proof}
By way of contradiction, suppose that the theorem fails.  Let $A$ be a set such that $\emptyset \not\subset A.$  From the above discussion, we can see that there exists an element $x$ such that $x \in \emptyset$ and $x \notin A$.  Let $x$ be such an element.  Since the emptyset has no elements, then $x \notin \emptyset$.  Thus, we have that $x \in \emptyset$ and $x \notin \emptyset.$ This contradiction proves that the theorem is true.
\end{proof}



\end{document}